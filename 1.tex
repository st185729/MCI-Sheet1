\documentclass[12pt]{scrartcl}

\input{../Style Template/Packages.tex}
\input{../Style Template/FormatAndHeader.tex}

\setcounter{sheetnr}{1}
\setcounter{enumi}{2}

\begin{document}

\exercise{}
    \begin{itemize}
        \item[\theenumi.1)]
            Utility beschreibt, ob ein Produkt seine Funktion erfolgreich erfüllt.
            Usability gibt an, inwieweit der Nutzer von dem Produkt bei der Benutzung unterstützt wird.
            Likability beschreibt nur, ob ein Produkt gemocht wird, unabhängig von seiner Effizienz und Nutzerfreundlichkeit.
        \item[\theenumi.2)]Lernbarkeit, Effizienz, Einprägsamkeit, Fehlerrate, Befriedigung 
    \end{itemize}

\setcounter{enumi}{4}
\exercise{}
\begin{itemize}
    \item[\theenumi.1)] Die unabhängige Variable ist der Diagrammtyp und die abhängige Variable ist die Ablesezeit.
    \item[\theenumi.2)] Die unabhängige Variable ist die Anzahl der Punkte in der Übung und die abhängige Variable ist die Note in der Klausur.
\end{itemize}

\stepcounter{enumi}
\exercise{}

\stepcounter{enumi}
\exercise{}
\begin{itemize}
    \item[\theenumi.1)] \, \\
        \begin{tabular}{c c c}
            Tag & Julia & Salome\\ \hline
            Mittwoch & & \\ \hline
            Donnerstag & & \\ \hline
            Freitag & & \\ \hline
            Samstag & & \\ \hline
            Sonntag & & \\ \hline
        \end{tabular}
\end{itemize}
        


\end{document}