\documentclass[12pt]{scrartcl}

\input{../Packages.tex}
\input{../FormatAndHeader.tex}

\setcounter{sheetnr}{1}
\setcounter{enumi}{2}

\begin{document}

\exercise{}
    \begin{itemize}
        \item[\theenumi.1)]
            Utility beschreibt, ob ein Produkt seine Funktion erfolgreich erfüllt.
            Usability gibt an, inwieweit der Nutzer von dem Produkt bei der Benutzung unterstützt wird.
            Likability beschreibt nur, ob ein Produkt gemocht wird, unabhängig von seiner Effizienz und Nutzerfreundlichkeit.
        \item[\theenumi.2)]Lernbarkeit, Effizienz, Einprägsamkeit, Fehlerrate, Befriedigung 
    \end{itemize}

\setcounter{enumi}{4}
\exercise{}
\begin{itemize}
    \item[\theenumi.1)] Die unabhängige Variable ist der Diagrammtyp und die abhängige Variable ist die Ablesezeit.
    \item[\theenumi.2)] Die unabhängige Variable ist die Anzahl der Punkte in der Übung und die abhängige Variable ist die Note in der Klausur.
\end{itemize}

\stepcounter{enumi}
\exercise{}

\begin{itemize}
    \item[\theenumi.1)] Latin square design: Es ist in dieser Studie sinnvoll, da durch die verschiedenen Reihenfolgen der Diagramme ausgeschlossen werden kann, dass die Ergebnisse von der gewählten Reihenfolge abhängt. Zudem kann die Anzahl der Teilnehmer geringer als bei einer within-Gruppenstudie gehalten werden, da nur 3 Sequenzen anstatt 6 Sequenzen durchlaufen und vorbereitet werden müssen.
	\item[\theenumi.2)] Unabhängige Variablen sind die Anzahl der Datenwerte (7,12,24)und die Art des Diagramms
	Unabhängige Variablen ist die Zeit
	\item[\theenumi.3)] Die Rückmeldung direkt nach der Eingabe kann ein Problem sein, weil dadurch die Teilnehmer entweder demotivierter oder selbstbewusster bzw. leichtsinniger bei der Antwortgabe werden.
	Ein mögliches Problem könnte sein, dass nur Probanden betrachtet wurden die einen hohen akademischen Abschluss haben. Während des Studiums kommt man mit dieser Art von Daten irgendwann in Kontakt, sodass man dort schon eine Art Strategie entwickelt.
\end{itemize}


\stepcounter{enumi}
\exercise{}
\begin{itemize}
    \item[\theenumi.1)] \, \\
        \begin{tabular}{c c c}
            Tag & Julia & Salome\\ \hline
            Mittwoch & & \\ \hline
            Donnerstag & & \\ \hline
            Freitag & & \\ \hline
            Samstag & & \\ \hline
            Sonntag & & \\ \hline
        \end{tabular}
\end{itemize}
        


\end{document}