\documentclass[12pt]{scrartcl}

% LaTeX Template für Abgaben an der Universität Stuttgart
% Autor: Sandro Speth
% Bei Fragen: Sandro.Speth@iste.uni-stuttgart.de
%-----------------------------------------------------------
% Modul fuer verwendete Pakete.
% Neue Pakete einfach einfuegen mit dem \usepackage Befehl:
% \usepackage[options]{packagename}
\usepackage[utf8]{inputenc}
\usepackage[T1]{fontenc}
\usepackage[ngerman]{babel}
\usepackage{lmodern}
\usepackage{graphicx}
\usepackage[pdftex,hyperref,dvipsnames]{xcolor}
\usepackage{listings}
\usepackage[a4paper,lmargin={2cm},rmargin={2cm},tmargin={3.5cm},bmargin = {2.5cm},headheight = {4cm}]{geometry}
\usepackage{amsmath,amssymb,amstext,amsthm}
\usepackage[lined,algonl,boxed]{algorithm2e}
% alternative zu algorithm2e:
%\usepackage[]{algorithm} %counter mit chapter
%\usepackage{algpseudocode}
\usepackage{tikz}
\usepackage{hyperref}
\usepackage{url}
\usepackage[inline]{enumitem} % Ermöglicht ändern der enum Item Zahlen
\usepackage[headsepline]{scrlayer-scrpage} 
\pagestyle{scrheadings} 
\usetikzlibrary{automata,positioning}
\usepackage{bookmark}
\usepackage{pgfplots}
% LaTeX Template für Abgaben an der Universität Stuttgart
% Autor: Sandro Speth
% Bei Fragen: Sandro.Speth@iste.uni-stuttgart.de
%-----------------------------------------------------------
% Modul beinhaltet Befehl fuer Aufgabennummerierung,
% sowie die Header Informationen.

% Überschreibt enumerate Befehl, sodass 1. Ebene Items mit
\renewcommand{\theenumi}{\Alph{enumi}}
% (a), (b), etc. nummeriert werden.
\renewcommand{\labelenumi}{\text{\theenumi}}

\renewcommand{\theenumii}{\arabic{enumii}}
% (a), (b), etc. nummeriert werden.
\renewcommand{\labelenumii}{\text{\theenumi.\theenumii}}


% Counter für das Blatt und die Aufgabennummer.
% Ersetze die Nummer des Übungsblattes und die Nummer der Aufgabe
% den Anforderungen entsprechend.
% Gesetz werden die counter in der hauptdatei, damit siese hier nicht jedes mal verändert werden muss
% Beachte:
% \setcounter{countername}{number}: Legt den Wert des Counters fest
% \stepcounter{countername}: Erhöht den Wert des Counters um 1.
\newcounter{sheetnr}
\newcounter{exnum}

% Befehl für die Aufgabentitel
\newcommand{\exercise}[1]{\section*{Aufgabe \theenumi )}} % Befehl für Aufgabentitel

% Formatierung der Kopfzeile
% \ohead: Setzt rechten Teil der Kopfzeile mit
% Namen und Matrikelnummern aller Bearbeiter
\ohead{Salome Knihs (3625589), Informatik Lehramt, Bachelor \\
Julia Waedt (3654521), Informatik, Bachelor}
% \chead{} kann mittleren Kopfzeilen Teil sezten
% \ihead: Setzt linken Teil der Kopfzeile mit
% Modulnamen, Semester und Übungsblattnummer
\ihead{Mensch-Computer-Interaktion\\
Sommersemester 2024\\
Übungsblatt \thesheetnr}

\setcounter{sheetnr}{1}
\setcounter{enumi}{2}

\begin{document}

\exercise{}
    \begin{itemize}
        \item[B.1)]
            Utility beschreibt, ob ein Produkt seine Funktion erfolgreich erfüllt.
            Usability gibt an, inwieweit der Nutzer von dem Produkt bei der Benutzung unterstützt wird.
            Likability beschreibt nur, ob ein Produkt gemocht wird, unabhängig von seiner Effizienz und Nutzerfreundlichkeit.
        \item[B.2)]Lernbarkeit, Effizienz, Einprägsamkeit, Fehlerrate, Befriedigung 
    \end{itemize}

\setcounter{enumi}{4}
\exercise{}
\begin{itemize}
    \item[D.1)] Die unabhängige Variable ist der Diagrammtyp und die abhängige Variable ist die Ablesezeit.
    \item[D.2)] Die unabhängige Variable ist die Anzahl der Punkte in der Übung und die abhängige Variable ist die Note in der Klausur.
\end{itemize}

\stepcounter{enumi}
\exercise{}

\stepcounter{enumi}
\exercise{}
\begin{itemize}
    \item[F.1)] \, \\
        \begin{tabular}{c c c}
            Tag & Julia & Salome\\ \hline
            Mittwoch & & \\ \hline
            Donnerstag & & \\ \hline
            Freitag & & \\ \hline
            Samstag & & \\ \hline
            Sonntag & & \\ \hline
        \end{tabular}
\end{itemize}
        


\end{document}