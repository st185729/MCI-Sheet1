\documentclass[12pt]{scrartcl}

% LaTeX Template für Abgaben an der Universität Stuttgart
% Autor: Sandro Speth
% Bei Fragen: Sandro.Speth@iste.uni-stuttgart.de
%-----------------------------------------------------------
% Modul fuer verwendete Pakete.
% Neue Pakete einfach einfuegen mit dem \usepackage Befehl:
% \usepackage[options]{packagename}
\usepackage[utf8]{inputenc}
\usepackage[T1]{fontenc}
\usepackage[ngerman]{babel}
\usepackage{lmodern}
\usepackage{graphicx}
\usepackage[pdftex,hyperref,dvipsnames]{xcolor}
\usepackage{listings}
\usepackage[a4paper,lmargin={2cm},rmargin={2cm},tmargin={3.5cm},bmargin = {2.5cm},headheight = {4cm}]{geometry}
\usepackage{amsmath,amssymb,amstext,amsthm}
\usepackage[lined,algonl,boxed]{algorithm2e}
% alternative zu algorithm2e:
%\usepackage[]{algorithm} %counter mit chapter
%\usepackage{algpseudocode}
\usepackage{tikz}
\usepackage{hyperref}
\usepackage{url}
\usepackage[inline]{enumitem} % Ermöglicht ändern der enum Item Zahlen
\usepackage[headsepline]{scrlayer-scrpage} 
\pagestyle{scrheadings} 
\usetikzlibrary{automata,positioning}
\usepackage{bookmark}
\usepackage{pgfplots}
% LaTeX Template für Abgaben an der Universität Stuttgart
% Autor: Sandro Speth
% Bei Fragen: Sandro.Speth@iste.uni-stuttgart.de
%-----------------------------------------------------------
% Modul beinhaltet Befehl fuer Aufgabennummerierung,
% sowie die Header Informationen.

% Überschreibt enumerate Befehl, sodass 1. Ebene Items mit
\renewcommand{\theenumi}{\Alph{enumi}}
% (a), (b), etc. nummeriert werden.
\renewcommand{\labelenumi}{\text{\theenumi}}

\renewcommand{\theenumii}{\arabic{enumii}}
% (a), (b), etc. nummeriert werden.
\renewcommand{\labelenumii}{\text{\theenumi.\theenumii}}


% Counter für das Blatt und die Aufgabennummer.
% Ersetze die Nummer des Übungsblattes und die Nummer der Aufgabe
% den Anforderungen entsprechend.
% Gesetz werden die counter in der hauptdatei, damit siese hier nicht jedes mal verändert werden muss
% Beachte:
% \setcounter{countername}{number}: Legt den Wert des Counters fest
% \stepcounter{countername}: Erhöht den Wert des Counters um 1.
\newcounter{sheetnr}
\newcounter{exnum}

% Befehl für die Aufgabentitel
\newcommand{\exercise}[1]{\section*{Aufgabe \theenumi )}} % Befehl für Aufgabentitel

% Formatierung der Kopfzeile
% \ohead: Setzt rechten Teil der Kopfzeile mit
% Namen und Matrikelnummern aller Bearbeiter
\ohead{Salome Knihs (3625589), Informatik Lehramt, Bachelor \\
Julia Waedt (3654521), Informatik, Bachelor}
% \chead{} kann mittleren Kopfzeilen Teil sezten
% \ihead: Setzt linken Teil der Kopfzeile mit
% Modulnamen, Semester und Übungsblattnummer
\ihead{Mensch-Computer-Interaktion\\
Sommersemester 2024\\
Übungsblatt \thesheetnr}

%% LaTeX Template für Abgaben an der Universität Stuttgart
% Autor: Sandro Speth
% Bei Fragen: Sandro.Speth@iste.uni-stuttgart.de
%-----------------------------------------------------------
% Modul fuer verwendete Pakete.
% Neue Pakete einfach einfuegen mit dem \usepackage Befehl:
% \usepackage[options]{packagename}
\usepackage[utf8]{inputenc}
\usepackage[T1]{fontenc}
\usepackage[ngerman]{babel}
\usepackage{lmodern}
\usepackage{graphicx}
\usepackage[pdftex,hyperref,dvipsnames]{xcolor}
\usepackage{listings}
\usepackage[a4paper,lmargin={2cm},rmargin={2cm},tmargin={3.5cm},bmargin = {2.5cm},headheight = {4cm}]{geometry}
\usepackage{amsmath,amssymb,amstext,amsthm}
\usepackage[lined,algonl,boxed]{algorithm2e}
% alternative zu algorithm2e:
%\usepackage[]{algorithm} %counter mit chapter
%\usepackage{algpseudocode}
\usepackage{tikz}
\usepackage{hyperref}
\usepackage{url}
\usepackage[inline]{enumitem} % Ermöglicht ändern der enum Item Zahlen
\usepackage[headsepline]{scrlayer-scrpage} 
\pagestyle{scrheadings} 
\usetikzlibrary{automata,positioning}
\usepackage{bookmark}
\usepackage{pgfplots}
%% LaTeX Template für Abgaben an der Universität Stuttgart
% Autor: Sandro Speth
% Bei Fragen: Sandro.Speth@iste.uni-stuttgart.de
%-----------------------------------------------------------
% Modul beinhaltet Befehl fuer Aufgabennummerierung,
% sowie die Header Informationen.

% Überschreibt enumerate Befehl, sodass 1. Ebene Items mit
\renewcommand{\theenumi}{\Alph{enumi}}
% (a), (b), etc. nummeriert werden.
\renewcommand{\labelenumi}{\text{\theenumi}}

\renewcommand{\theenumii}{\arabic{enumii}}
% (a), (b), etc. nummeriert werden.
\renewcommand{\labelenumii}{\text{\theenumi.\theenumii}}


% Counter für das Blatt und die Aufgabennummer.
% Ersetze die Nummer des Übungsblattes und die Nummer der Aufgabe
% den Anforderungen entsprechend.
% Gesetz werden die counter in der hauptdatei, damit siese hier nicht jedes mal verändert werden muss
% Beachte:
% \setcounter{countername}{number}: Legt den Wert des Counters fest
% \stepcounter{countername}: Erhöht den Wert des Counters um 1.
\newcounter{sheetnr}
\newcounter{exnum}

% Befehl für die Aufgabentitel
\newcommand{\exercise}[1]{\section*{Aufgabe \theenumi )}} % Befehl für Aufgabentitel

% Formatierung der Kopfzeile
% \ohead: Setzt rechten Teil der Kopfzeile mit
% Namen und Matrikelnummern aller Bearbeiter
\ohead{Salome Knihs (3625589), Informatik Lehramt, Bachelor \\
Julia Waedt (3654521), Informatik, Bachelor}
% \chead{} kann mittleren Kopfzeilen Teil sezten
% \ihead: Setzt linken Teil der Kopfzeile mit
% Modulnamen, Semester und Übungsblattnummer
\ihead{Mensch-Computer-Interaktion\\
Sommersemester 2024\\
Übungsblatt \thesheetnr}

\setcounter{sheetnr}{1}
\setcounter{enumi}{2}

\begin{document}

\exercise{}
    \begin{itemize}
        \item[\theenumi.1)]
            Utility beschreibt, ob ein Produkt seine Funktion erfolgreich erfüllt.
            Usability gibt an, inwieweit der Nutzer von dem Produkt bei der Benutzung unterstützt wird.
            Likability beschreibt nur, ob ein Produkt gemocht wird, unabhängig von seiner Effizienz und Nutzerfreundlichkeit.
        \item[\theenumi.2)]Lernbarkeit, Effizienz, Einprägsamkeit, Fehlerrate, Befriedigung 
    \end{itemize}
\newpage

\setcounter{enumi}{4}
\exercise{}
\begin{itemize}
    \item[\theenumi.1)] Die unabhängige Variable ist der Diagrammtyp und die abhängige Variable ist die Ablesezeit.
    \item[\theenumi.2)] Die unabhängige Variable ist die Anzahl der Punkte in der Übung und die abhängige Variable ist die Note in der Klausur.
\end{itemize}
\newpage

\stepcounter{enumi}
\exercise{}
\begin{itemize}
    \item[\theenumi.1)] Latin square design: Es ist in dieser Studie sinnvoll, da durch die verschiedenen Reihenfolgen der Diagramme 
                        ausgeschlossen werden kann, dass die Ergebnisse von der gewählten Reihenfolge abhängen. Zudem kann die Anzahl der Teilnehmer geringer als bei einer within-Gruppenstudie gehalten werden, 
                        da nur 3 Sequenzen anstatt 6 Sequenzen durchlaufen und vorbereitet werden müssen.
	\item[\theenumi.2)] Unabhängigen Variablen sind die Anzahl der Datenwerte (7,12,24) und die Art der Diagramme. Die abhängige Variable ist die Zeit, die die Teilnehmer jeweils gebraucht haben.
	\item[\theenumi.3)] Die Rückmeldung direkt nach der Eingabe kann ein Problem sein, weil dadurch die Teilnehmer entweder demotivierter oder selbstbewusster bzw. leichtsinniger bei der Antwortgabe werden.
	Ein weiteres mögliches Problem könnte sein, dass nur Probanden betrachtet wurden, die einen hohen akademischen Abschluss haben. Sie sind dadurch schon vertrauter mit Diagrammen aller Art.
\end{itemize}
\newpage

\stepcounter{enumi}
\exercise{}
\begin{itemize}
    \item[\theenumi.1)] \, \\
        \begin{tabular}{c c c}
            \textbf{Tag} & \textbf{Julia} & \textbf{Salome}\\ \hline
            Mittwoch & 390 Min & 600 Min\\ \hline
            Donnerstag & 720 Min & 630 Min\\ \hline
            Freitag & 0 Min & 390 Min\\ \hline
            Samstag & 0 Min & 30 Min\\ \hline
            Sonntag & 0 Min & 330 Min\\ \hline
        \end{tabular}
    \item[\theenumi.2)] Durchschnitt Julia: 222 Min\\
                        Durchschnitt Salome: 396 Min
    \item[\theenumi.3)] \, \\
        \begin{tabular}{c | c c}
            & \textbf{Julia} & \textbf{Salome}\\ \hline
            \textbf{Durchschnitt} & 222 Min& 396 Min\\\hline
            \textbf{Standardabweichung} & 291.232 Min & 277.013 Min \\\hline
            \textbf{Median} & 0 Min & 390 Min
        \end{tabular}
    \item[\theenumi.4)] \, \\

    \begin{tikzpicture}
    \begin{axis}[
        width  = 0.85*\textwidth,
        height = 8cm,
        major x tick style = transparent,
        ybar,
        bar width=14pt,
        ymajorgrids = true,
        ylabel = {Zeitaufwand},
        symbolic x coords={Mi, Do, Fr, Sa, So},
        xtick = data,
        scaled y ticks = false,
    ]
      \addplot [ybar, fill = orange] coordinates {(Mi, 390)(Do, 720) (Fr, 0) (Sa, 0) (So, 0)};
      \addplot [ybar, fill = blue] coordinates {(Mi, 600)(Do, 630)(Fr, 390)(Sa, 30)(So, 330)};

      \legend{Julia, Salome}
      \end{axis}
    \end{tikzpicture}
\end{itemize}
\newpage

\stepcounter{enumi}
\exercise{}
\begin{itemize}
    \item[\theenumi.1)] \textbf{Nullhypothese $H_0$:} Die Jahresdurchschnitstemperatur in Deutschland hat sich 
                                                      im Zeitraum von 1991 bis 2022 nicht verändert.\\
                        \textbf{Durchschnittstemperatur (1961 bis 1990):} $\mu$ = 8.2 °C\\
                        \textbf{Durchschnittstemperatur (Stichprobe):} $\overline{x}$ = 9.3375 °C\\
                        \textbf{Standardabweichung:} s $\approx$ 0,732 °C\\
                        \textbf{Stichprobengröße:} n = 32\\
                        \textbf{Anzahl Freiheitsgrade:} df = 31\\
                        \textbf{Berechneter t-Wert:} \\\\
                        $t = \dfrac{\overline{x}-\mu}{\dfrac{s}{\sqrt{n}}} = \dfrac{9.3375-8.2}{\dfrac{0.732}{\sqrt{32}}} \approx 8.791$\\\\
                        \textbf{Kritischer t-Wert:} 2.453\\
                        Der berechnete t-Wert ist größer als der kritische t-Wert. Das bedeutet, die Nullhypothese wird verworfen und man geht davon aus,
                        dass die Jahresdurchschnitstemperatur in Deutschland sich von 1991 bis 2022 verändert hat.
                        
	\item[\theenumi.2)] \textbf{Nullhypothese $H_0$:} Die deutsche Jahresdurchschnitstemperatur ist gleich hoch wie die globale Jahresdurchschnittstemperatur.\\
	Für die globalen Werte gilt:\\
	\textbf{Durchschnittstemperatur (1951 bis 1980):} $\mu$ = 8.79 °C\\
						\textbf{Durchschnittstemperatur (Stichprobe):} $\overline{x} = $9.429 °C\\
                        \textbf{Standardabweichung:} s $\approx$ 0.344 °C \\
                        \textbf{Stichprobengröße:} n = 35\\
                        \textbf{Anzahl Freiheitsgrade:} df = 34 \\
                        \\
\textbf{Berechneter t-Wert:}

$t=\dfrac{(\overline{x_{1}}-\overline{x_{2}}) - (\mu_{1}-\mu_{2})}{\sqrt{\dfrac{{s_{1}}^{2}}{n_1}+\dfrac{{s_{2}}^{2}}{n_2}}} 
=\dfrac{(9.3375 - 9.429)- (8.2-8.79)}{\sqrt{\dfrac{{0,732}^{2}}{32}+\dfrac{{0.344}^{2}}{35}}}  \approx 3,514$\\

\textbf{Anzahl Freiheitsgrade gesamt:} 31 + 34 = 65 \\
\textbf{Kritischer t-Wert:} 1.669\\
Der berechnete t-Wert ist größer als der kritische t-Wert, daher wird die Nullhypothese verworfen. Man geht also davon aus, dass die Jahresdurchschnittstemperatur in Deutschland und global signifikant unterschiedlich sind.
\end{itemize}
                        
                        


\end{document}