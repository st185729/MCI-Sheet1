\documentclass[12pt]{scrartcl}

% LaTeX Template für Abgaben an der Universität Stuttgart
% Autor: Sandro Speth
% Bei Fragen: Sandro.Speth@iste.uni-stuttgart.de
%-----------------------------------------------------------
% Modul fuer verwendete Pakete.
% Neue Pakete einfach einfuegen mit dem \usepackage Befehl:
% \usepackage[options]{packagename}
\usepackage[utf8]{inputenc}
\usepackage[T1]{fontenc}
\usepackage[ngerman]{babel}
\usepackage{lmodern}
\usepackage{graphicx}
\usepackage[pdftex,hyperref,dvipsnames]{xcolor}
\usepackage{listings}
\usepackage[a4paper,lmargin={2cm},rmargin={2cm},tmargin={3.5cm},bmargin = {2.5cm},headheight = {4cm}]{geometry}
\usepackage{amsmath,amssymb,amstext,amsthm}
\usepackage[lined,algonl,boxed]{algorithm2e}
% alternative zu algorithm2e:
%\usepackage[]{algorithm} %counter mit chapter
%\usepackage{algpseudocode}
\usepackage{tikz}
\usepackage{hyperref}
\usepackage{url}
\usepackage[inline]{enumitem} % Ermöglicht ändern der enum Item Zahlen
\usepackage[headsepline]{scrlayer-scrpage} 
\pagestyle{scrheadings} 
\usetikzlibrary{automata,positioning}
\usepackage{bookmark}
\usepackage{pgfplots}
% LaTeX Template für Abgaben an der Universität Stuttgart
% Autor: Sandro Speth
% Bei Fragen: Sandro.Speth@iste.uni-stuttgart.de
%-----------------------------------------------------------
% Modul beinhaltet Befehl fuer Aufgabennummerierung,
% sowie die Header Informationen.

% Überschreibt enumerate Befehl, sodass 1. Ebene Items mit
\renewcommand{\theenumi}{\Alph{enumi}}
% (a), (b), etc. nummeriert werden.
\renewcommand{\labelenumi}{\text{\theenumi}}

\renewcommand{\theenumii}{\arabic{enumii}}
% (a), (b), etc. nummeriert werden.
\renewcommand{\labelenumii}{\text{\theenumi.\theenumii}}


% Counter für das Blatt und die Aufgabennummer.
% Ersetze die Nummer des Übungsblattes und die Nummer der Aufgabe
% den Anforderungen entsprechend.
% Gesetz werden die counter in der hauptdatei, damit siese hier nicht jedes mal verändert werden muss
% Beachte:
% \setcounter{countername}{number}: Legt den Wert des Counters fest
% \stepcounter{countername}: Erhöht den Wert des Counters um 1.
\newcounter{sheetnr}
\newcounter{exnum}

% Befehl für die Aufgabentitel
\newcommand{\exercise}[1]{\section*{Aufgabe \theenumi )}} % Befehl für Aufgabentitel

% Formatierung der Kopfzeile
% \ohead: Setzt rechten Teil der Kopfzeile mit
% Namen und Matrikelnummern aller Bearbeiter
\ohead{Julia Waedt (3654521)}
\chead{Individual}
% \ihead: Setzt linken Teil der Kopfzeile mit
% Modulnamen, Semester und Übungsblattnummer
\ihead{Mensch-Computer-Interaktion\\
Sommersemester 2024\\
Übungsblatt \thesheetnr}

\setcounter{sheetnr}{1}
\setcounter{enumi}{1}

\begin{document}

\exercise{}
\begin{itemize}
    \item[\theenumi.1)] \begin{itemize}
                            \item[$\bullet$] Ein Beispiel sind Fahrräder: Sie sollten zunächst nur zur Fortbewegung dienen, wurden jedoch bald auch als Sportgeräte verwendet. 
                                             Nun gab es für die unterschiedlichen Sportarten verschiedene neue Anforderungen an das Fahrrad, weshalb neue, spezialisierte Fahrradtypen entwickelt wurden. 
                                             Hierzu zählen zum Beispiel Rennräder oder Mountainbikes. Diese werden immer weiter optimiert, um noch besser den Ansprüchen der Radsportler zu entsprechen.
                            \item[$\bullet$] Die Hundezucht ist ein weiteres Beispiel. Wie genau die Domestizierung der Wölfe abgelaufen ist, ist nicht klar, jedoch ist bekannt,
                                             das daraus viele neue Anforderungen an die Tiere entstanden sind. Es wurden daraufhin neue Rassen gezüchtet, die den neuen Ansprüchen entsprachen,
                                             zum Beispiel Schäfer-, Jagd- oder Haushunde. Heutzutage gibts es teilweise wieder neue Bedürfnisse, wie das Vermeiden von Qualzüchtungen. Das führt dazu, dass
                                             einige historische Züchtungen, wie zum Besipiel der Mops, wieder abgeändert werden, um den Anforderungen der heutigen Zeit gerecht zu werden. So entstand der Retro-Mops.
                        \end{itemize}
    \item[\theenumi.2)] Ein neues Bedürfnis wird wahrscheinlich sein, dass das Mixed-Reality-Gerät kleiner sein und das Ein- und Abschalten der Mixed-Reality einfacher ablaufen sollte.
                        Die Brillen würden dann also immer kleiner werden und somit auch einfacher auf- und abzusetzen. Im Extremfall könnte es irgendwann nur noch ein implantierter Chip werden, der
                        wie durch Knopfdruck die Mixed-Reality aktivieren oder abschalten kann.
\end{itemize}

\setcounter{enumi}{3}
\exercise{}
User Experience ist alles, was Benutzer bei der Nutzung eines Produktes wahrnehmen und welche Emotionen es in ihnen auslöst.
Dies hängt nicht unbedingt direkt mit Usability oder Utility zusammen. Das heißt, Produkte müssen nicht 
zwingend eine optimale Nutzbarkeit oder Effizienz haben, um eine gute User Experience zu erzeugen. Ein Beispiel 
hierfür ist die im Video vorgestellte Lampe. Sie imitiert einen Sonnenaufgang, um so eine besonders angenehme Art aufzuwachen zu bieten.
Die Lampe selbst sieht eher unscheinbar aus und sie hat vielleicht auch nicht die effektivste Weckfunktion verglichen mit anderen Weckern, dafür erzeugt sie 
aber eine besondere Erfahrung, welche kein anderer Wecker mit sich bringt.

\end{document}